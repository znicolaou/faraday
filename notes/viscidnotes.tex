\documentclass[aps,pre,amsmath,amssymb,floatfix,onecolumn,notitlepage,10pt]{revtex4-1}
\usepackage{graphicx,isomath,bm}
\usepackage[colorlinks=true,linkcolor=blue,citecolor=blue,urlcolor=blue,pdfborderstyle={/S/U/W 1}]{hyperref}

\begin{document}

\title{Wave dispersion in viscid Faraday waves with periodic heterogeneity}
\author{Zachary G. Nicolaou}
\date{\today}

\maketitle
\tableofcontents

\section{Introduction}

\section{Viscid fluids with periodic heterogeneity}
The incompressible Navier-Stokes equations govern the fluid motion,
\begin{align}
\rho \left(\frac{\partial \mathbf{u}}{\partial t} + (\mathbf{u}.\bm{\nabla})\mathbf{u}\right) - \mu \nabla^2 \mathbf{u} &= -\bm{\nabla}P + \rho \mathbf{g}, \label{ns} \\
\bm{\nabla} \cdot \mathbf{u} &= 0. \label{incompressible}
\end{align}
Rotational symmetry is broken by the gravitational body force, which is assumed to be oriented in the vertical direction $\hat{\mathbf{z}}$. Accordingly, we separate the lateral coordinates into a vector $\mathbf{x}=(x,y)$ and integrate out the $z$ dependence in the following. Furthermore, we take an accelerated frame of reference that moves with the vertical vibration of the container, which implies a periodic body force $\rho \mathbf{g} = \hat{\mathbf{z}}\rho (g + a_d\cos(\omega_d t))$, where $g$ is the gravitational acceleration, $a_d$ is the driving acceleration, and $\omega_d$ is the driving frequency.

The fluid-air interface is a free surface, occuring at $z=h_0 + H(\mathbf{x})$, where $h_0$ is the unperturbed fluid depth. The motion of the fluid surface is governed by a kinematic boundary condition
\begin{equation}
\left. \frac{\partial h}{\partial t} + \mathbf{u}\cdot\nabla h - \mathbf{u}\cdot\hat{\mathbf{z}} \right\rvert_{z=h_0+H} = 0. \label{kinematic}
\end{equation}
The tangential and normal  stress-balance conditions at the free interface ensure conservation of momentum
\begin{align}
\left. 2\mu\hat{\mathbf{n}}\cdot E \cdot \hat{\mathbf{t}}_x  \right\rvert_{z=h_0+H} = 0,  \label{tangentialstress1} \\
\left. 2\mu\hat{\mathbf{n}}\cdot E \cdot \hat{\mathbf{t}}_y  \right\rvert_{z=h_0+H}  = 0, \label{tangentialstress2} \\
\left. P - 2\mu \hat{\mathbf{n}} \cdot E \cdot \hat{\mathbf{n}} - \sigma \bm{\nabla}_s \cdot \hat{\mathbf{n}}  \right\rvert_{z=h_0+H} &= 0m, \label{normalstress}
\end{align}
where $\hat{\mathbf{n}}$ %= \frac{\hat{\mathbf{z}} - \hat{\mathbf{x}}\frac{\partial h}{\partial x} - \hat{\mathbf{y}}\frac{\partial h}{\partial y}}{\sqrt{1+(\frac{\partial h}{\partial x})^2+(\frac{\partial h}{\partial y})^2}}$
is the outward-pointing normal vector to the free surface, $\mathbf{t}_x$ and $\mathbf{t}_y$ are the tangential surface directions, $\sigma$ is the surface tension, $\bm{\nabla}_s$ is the surface gradient operator, and $E$ is symmetric the rate of strain tensor, with Cartesian component $E_{ij} = \frac{1}{2}\left(\frac{\partial u_{i}}{\partial x_j} + \frac{\partial u_{j}}{\partial x_i}\right)$.
Finally, there are no-slip and no-penetration boundary conditions on the fluid-solid interface, which occurs at $z=h_s(\mathbf{x})$,
\begin{equation}
\left. \mathbf{u} \right\rvert_{z=h_s} = \mathbf{0}. \label{noslip}
\end{equation}

We seek to study the linear stability of the motionless base state $\mathbf{u}_0 = \mathbf{0}$, $P_0 = \rho(g+a_d\cos(\omega_d t))z$, $H_0=0$. Let $\mathbf{u} = \mathbf{u}_0 + \bar{\mathbf{u}}$, $P=P_0+\bar{P}$, and $H=H_0+\bar{H}$, where $\bar{\mathbf{u}} = (u,v,w)$, $\bar{P}$, and $\bar{H}$ are small.  We assume that the solid interface $h_s(\mathbf{x})$ is periodic with respect to two translations, $h_s(\mathbf{x}) = h_s(\mathbf{x}+\mathbf{r}_1)$ and $h_s(\mathbf{x}) = h_s(\mathbf{x}+\mathbf{r}_2)$. Given the periodicity in space and time, we know that the linearized operator that we seek will commute with the discrete space and time translation operations. This implies that we can expand the perturbations into modes using a Floquet ansatz
\begin{align}
u &= e^{i\mathbf{k}.\mathbf{x} + i\omega t}\sum_{lmn} u_{lmn}(z) e^{i(m\mathbf{k}_1\cdot \mathbf{x} + n\mathbf{k}_2\cdot \mathbf{x} + l\omega_d t)}, \label{floq1} \\
v &= e^{i\mathbf{k}.\mathbf{x} + s t}\sum_{lmn} v_{lmn}(z) e^{i(m\mathbf{k}_1\cdot \mathbf{x} + n\mathbf{k}_2\cdot \mathbf{x} + l\omega_d t)}, \label{floq2} \\
w &= e^{i\mathbf{k}.\mathbf{x} + i\omega t}\sum_{lmn} w_{lmn}(z) e^{i(m\mathbf{k}_1\cdot \mathbf{x} + n\mathbf{k}_2\cdot \mathbf{x} + l\omega_d t)}, \label{floq3} \\
\bar{P} &= e^{i\mathbf{k}.\mathbf{x} + i\omega t}\sum_{lmn} P_{lmn}(z) e^{i(m\mathbf{k}_1\cdot \mathbf{x} + n\mathbf{k}_2\cdot \mathbf{x} + l\omega_d t)}, \label{floq4} \\
\bar{H} &= e^{i\mathbf{k}.\mathbf{x} + i\omega t}\sum_{lmn} H_{lmn} e^{i(m\mathbf{k}_1\cdot \mathbf{x} + n\mathbf{k}_2\cdot \mathbf{x} + l\omega_d t)},  \label{floq5}
\end{align}
where the reciprocal lattice vectors $\mathbf{k}_i$ are defined as the dual basis to the discrete translational vectors $\mathbf{r}_j \cdot \mathbf{k}_i = 2\pi\delta_{ij}$. We seek to determine the relationship between the wavevector $\mathbf{k}$ and the (complex) response frequency $\omega$, which we refer to as the dispersion relation. (Note about the periodicity in space and time and the first Brillouin zones).

We now seek to eliminate the $z$ dependence on the Floquet mode amplitudes $u_{lmn}(z)$, $v_{lmn}(z)$, $w_{lmn}(z)$, and $P_{lmn}(z)$. Note first that the advective term $\mathbf{u}\cdot\bm{\nabla}\mathbf{u}$ in Eq.~\eqref{ns} is negligible in the linear regime, since $\mathbf{u}=\bar{\mathbf{u}}$ is small. Then, taking the divergence of Eq.~\eqref{ns} and applying Eq.~\eqref{incompressible}, it follows that
\begin{equation}
\nabla^2 \bar{P}=0.
\end{equation}
Inserting Eq.~\eqref{floq4} and projecting onto the modes (by multiplying by $e^{-i\mathbf{k}\cdot x-st-i(m'\mathbf{k}_1\cdot x+n'\mathbf{k}_2\cdot x+l'\omega_d t)}$ and integrating over the two-dimensional spatial unit cell $V$ spanned by $\mathbf{r}_1$ and $\mathbf{r}_2$), it follows that
\begin{equation}
\frac{\partial^2 P_{lmn}}{\partial z^2} - \left |\bm{\kappa}_{mn}\right|^2 P_{lmn} = 0,
\end{equation}
where $\bm{\kappa}_{mn} = \mathbf{k} + m\mathbf{k}_1 + n\mathbf{k}_2$, and thus
\begin{equation}
P_{lmn} = A_{lmn}e^{|\bm{\kappa}_{mn}|z} + B_{lmn}e^{-|\bm{\kappa}_{mn}|z}. \label{Psol}
\end{equation}
Next, inserting Eq.~\eqref{Psol} back into Eq.~\eqref{ns} and projecting onto the modes, we find
\begin{align}
\rho\Omega_{lmn} u_{lmn} - \mu\frac{\partial ^2 u_{lmn}}{\partial z^2} &= -i\bm{\kappa}_{mn}\cdot \hat{x} (A_{lmn}e^{|\bm{\kappa}_{mn}|z} + B_{lmn}e^{-|\bm{\kappa}_{mn}|z}), \\
\rho\Omega_{lmn} v_{lmn}  - \mu\frac{\partial ^2 v_{lmn}}{\partial z^2} &= -i\bm{\kappa}_{mn}\cdot \hat{y} (A_{lmn}e^{|\bm{\kappa}_{mn}|z} + B_{lmn}e^{-|\bm{\kappa}_{mn}|z}), \\
\rho\Omega_{lmn} w_{lmn}  - \mu\frac{\partial ^2 w_{lmn}}{\partial z^2} &= -|\bm{\kappa}_{mn}| (A_{lmn}e^{|\bm{\kappa}_{mn}|z} - B_{lmn}e^{-|\bm{\kappa}_{mn}|z}),
\end{align}
where $\Omega_{lmn} = i(\omega+l \omega_d) + \frac{\mu}{\rho} |\bm{\kappa}_{mn}|^2$. Thus,
\begin{align}
u_{lmn}(z) &= A^x_{lmn} e^{\sqrt{\rho\Omega_{lmn}/\mu}z} + B^x_{lmn} e^{-\sqrt{\rho\Omega_{lmn}/\mu}z} - \frac{\bm{\kappa}_{mn}\cdot \hat{x}}{\rho(\omega+l\omega_d)}(A_{lmn}e^{|\bm{\kappa}_{mn}|z} + B_{lmn}e^{-|\bm{\kappa}_{mn}|z}), \label{usol} \\
v_{lmn}(z) &= A^y_{lmn} e^{\sqrt{\rho\Omega_{lmn}/\mu}z} + B^y_{lmn} e^{-\sqrt{\rho\Omega_{lmn}/\mu}z} - \frac{\bm{\kappa}_{mn}\cdot \hat{y}}{\rho(\omega+l\omega_d)}(A_{lmn}e^{|\bm{\kappa}_{mn}|z} + B_{lmn}e^{-|\bm{\kappa}_{mn}|z}), \label{vsol} \\
w_{lmn}(z) &= A^z_{lmn} e^{\sqrt{\rho\Omega_{lmn}/\mu}z} + B^z_{lmn} e^{-\sqrt{\rho\Omega_{lmn}/\mu}z} + \frac{i|\bm{\kappa}_{mn}|}{\rho(\omega+l\omega_d)}(A_{lmn}e^{|\bm{\kappa}_{mn}|z} - B_{lmn}e^{-|\bm{\kappa}_{mn}|z}). \label{wsol}
\end{align}
To complete the solution to the bulk fluid equations, we substitute Eq.~\eqref{usol}-\eqref{wsol} into the continuity equation, leading to two constraints,
\begin{align}
i\bm{\kappa}_{mn}\cdot\hat{\mathbf{x}}A^x_{lmn}+i\bm{\kappa}_{mn}\cdot\hat{\mathbf{y}}A^y_{lmn}+\sqrt{\rho \Omega_{lmn}/\mu}A^z_{lmn} &= 0, \label{continuitysol1} \\
i\bm{\kappa}_{mn}\cdot\hat{\mathbf{x}}B^x_{lmn}+i\bm{\kappa}_{mn}\cdot\hat{\mathbf{y}}B^y_{lmn}-\sqrt{\rho \Omega_{lmn}/\mu}B^z_{lmn} &= 0. \label{continuitysol2}
\end{align}

We have thus eliminated the $z$ dependence from the linearized equations and are left with nine constants per mode $\bm{\xi}_{lmn}=(A_{lmn}, B_{lmn}, A^x_{lmb}, B^x_{lmb},  A^y_{lmb}, B^y_{lmb},  A^z_{lmb}, B^z_{lmb}, H_{lmn})$, which we must eliminate with the remaining equations. There are seven boundary conditions left to apply in Eq.~\eqref{noslip}-\eqref{tangentialstress2}, along with the two constraints in Eqs.~\eqref{continuitysol1}-\eqref{continuitysol2}, which form the linear system.  Denoting the $k$th component of $\bm{\xi}_{lmn}$ as $\xi_{klmn}$, we can express this system through a matrix of elements $A^{klmn}_{k'l'm'n'}(\mathbf{k}, \omega)$ as
\begin{equation}
\sum_{klmn} A^{klmn}_{k'l'm'n'}(\mathbf{k}, \omega) \xi_{klmn} = 0. \label{linear}
\end{equation}
We encode Eqs.~\eqref{continuitysol1}-\eqref{continuitysol2} for the first two sets of rows in $A^{klmn}_{k'l'm'n'}(\mathbf{k}, \omega)$ with
\begin{align}
A^{klmn}_{1l'm'n'}(\mathbf{k}, \omega) &= (i\bm{\kappa}_{mn}\cdot\hat{\mathbf{x}}\delta^{k}_{3}+i\bm{\kappa}_{mn}\cdot\hat{\mathbf{y}}\delta^{k}_{5}+\sqrt{\rho \Omega_{lmn}/\mu}\delta^{k}_{7})\delta^{lmn}_{l'm'n'}, \label{lconstraint1} \\
A^{klmn}_{2l'm'n'}(\mathbf{k}, \omega) &= (i\bm{\kappa}_{mn}\cdot\hat{\mathbf{x}}\delta^{k}_{4}+i\bm{\kappa}_{mn}\cdot\hat{\mathbf{y}}\delta^{k}_{6}-\sqrt{\rho \Omega_{lmn}/\mu}\delta^{k}_{8})\delta^{lmn}_{l'm'n'}. \label{lconstraint2}
\end{align}

We use the tangential stress balance conditions in Eqs.~\eqref{tangentialstress1}-\eqref{tangentialstress2} for the next two sets of rows. Since the symmetric rate of shear is already small,  we can use the zeroth order $\hat{\mathbf{n}}=\hat{\mathbf{z}}$, $\hat{\mathbf{t}}_x=\hat{\mathbf{x}}$, and $\hat{\mathbf{t}}_y=\hat{\mathbf{y}}$. Projecting onto the modes, we find
\begin{align}
A^{klmn}_{3l'm'n'}(\mathbf{k}, \omega) &=\Big[ e^{\sqrt{\rho\Omega_{lmn}/\mu}h_0}(\sqrt{\rho\Omega_{lmn}/\mu}\delta^k_3 + \bm{\kappa}_{mn}\cdot \hat{\mathbf{x}}\delta^k_7) + e^{-\sqrt{\rho\Omega_{lmn}/\mu}h_0}(-\sqrt{\rho\Omega_{lmn}/\mu}\delta^k_4 + \bm{\kappa}_{mn}\cdot \hat{\mathbf{x}}\delta^k_8) \nonumber \\
&\quad + 2|\bm{\kappa}_{mn}| \frac{\bm{\kappa}_{mn}\cdot \hat{\mathbf{x}}}{\rho(\omega + l \omega_d)} \left(-e^{|\bm{\kappa}_{mn}|h_0} \delta^k_1 + e^{-|\bm{\kappa}_{mn}|h_0} \delta^k_2 \right)\Big] \delta^{lmn}_{l'm'n'},   \\
A^{klmn}_{4l'm'n'}(\mathbf{k}, \omega)  &=  \Big[e^{\sqrt{\rho\Omega_{lmn}/\mu}h_0}(\sqrt{\rho\Omega_{lmn}/\mu}\delta^k_5 + \bm{\kappa}_{mn}\cdot \hat{\mathbf{y}}\delta^k_7) + e^{-\sqrt{\rho\Omega_{lmn}/\mu}h_0}(-\sqrt{\rho\Omega_{lmn}/\mu}\delta^k_6 + \bm{\kappa}_{mn}\cdot \hat{\mathbf{y}}\delta^k_8) \nonumber \\
&\quad + 2|\bm{\kappa}_{mn}| \frac{\bm{\kappa}_{mn}\cdot \hat{\mathbf{y}}}{\rho(\omega + l \omega_d)} \left(-e^{|\bm{\kappa}_{mn}|h_0} \delta^k_1 + e^{-|\bm{\kappa}_{mn}|h_0} \delta^k_2 \right)\Big] \delta^{lmn}_{l'm'n'}.
\end{align}

Next, we use the the normal stress balance conditions remains for the fifth set of row. The viscous term again follows from the zeroth order $\hat{\mathbf{n}}=\hat{z}$, while the leading order capillary term becomes $\sigma \nabla_s \cdot \hat{\mathbf{n}} = -\sigma \left(\frac{\partial^2 H}{\partial x^2}+\frac{\partial^2 H}{\partial y^2}\right)$. Since the periodic forcing term $a_d \cos(\omega_d t)$ appears in the pressure term, this equation is not diagonal in the $(l,l')$ indices,
\begin{align}
A^{klmn}_{5l'm'n'}(\mathbf{k}, \omega) &= \Big[ -\rho (g+\sigma|\bm{\kappa}_{nm}|^2)\delta^k_9 + \left(\rho+\frac{2\mu|\bm{\kappa}_{nm}|^2}{i\rho(\omega+l\omega_d)}\right)\left(e^{|\bm{\kappa}_{nm}|h_0}\delta^k_1 + e^{-|\bm{\kappa}_{nm}|h_0}\delta^k_2 \right) \nonumber \\
&\quad - 2\mu/\rho\sqrt{\rho\Omega_{lmn}/\mu}\left(e^{\sqrt{\rho\Omega_{lmn}/\mu}h_0}\delta^k_7 - e^{-\sqrt{\rho\Omega_{lmn}/\mu}h_0}\delta^k_8\right)\Big]\delta^{mnl}_{m'n'l'}-\rho a_d \delta^k_9 (\delta^l_{l'+1}+\delta^l_{l'-1})\delta^{mn}_{m'n'}/2.
\end{align}

We use the kinematic boundary condition Eq.~\eqref{kinematic} for the sixth set of rows, noting that $\mathbf{u}\cdot\nabla h$ can be neglected and projecting onto the modes
\begin{align}
A^{klmn}_{6l'm'n'}(\mathbf{k}, \omega) = \left[i(\omega+l\omega_d)\delta^k_9 - e^{\sqrt{\rho\Omega_{lmn}/\mu}h_0}\delta^k_7 - e^{-\sqrt{\rho\Omega_{lmn}/\mu}h_0} \delta^k_8 - \frac{i|\bm{\kappa}_{mn}|}{\rho(\omega+l\omega_d)}(e^{|\bm{\kappa}_{mn}|h_0}\delta^k_1 - e^{-|\bm{\kappa}_{mn}|h_0}\delta^k_2)\right]\delta^{lmn}_{l'm'n'}. \label{lkinematic}
\end{align}

The final three rows will follow from the no slip condition at the solid-liquid interface, Eq.~\eqref{noslip}. Unlike the first five rows, the heterogeneity in the substrate shape $h_s(\mathbf{x})$ means that the linearized equations will not be diagonal in the spatial indices $(n,n')$ and $(m,m')$. In order to project the equations onto the Floquet basis, we need to evaluate integrals $I_{n-n'\, m-m'}=\int_V e^{i(m-m')\mathbf{k}_1\cdot\mathbf{x}+i(n-n')\mathbf{k}_2\cdot\mathbf{x}}e^{\alpha h_s(\mathbf{x})} d^2x$, where $V$ is the unit cell spanned by the spatial periodicity vectors $\mathbf{r}_1$ and $\mathbf{r}_2$. Let us assume, for simplicity, that $h_s(\mathbf{x}) = a_s(f(\mathbf{k}_1\cdot x)+f(\mathbf{k}_2\cdot x))/2$, where $f$ is a $2\pi$-periodic function and $a_s$ is a parameter describing the substrate height. In this case, the relevant integral factorizes as $I_{n-n'\, m-m'} = I_{n-n'}(\alpha a_s/2)I_{m-m'}(\alpha a_s/2)$, where $I_{n-n'}=\int_0^{2\pi} e^{i\lambda+\alpha a_s f(\lambda)/2} d\lambda$. When $f(\lambda)=\sin(\lambda)$, for example, the $I_{n-n'}$ is the modified Bessel function. Then, projecting Eq.~\eqref{noslip} onto the modes gives
\begin{align}
A^{klmn}_{7l'm'n'}(\mathbf{k}, \omega) &= \Big[ \delta^k_{3} I_{n-n'\, m-m'}(\sqrt{\rho\Omega_{lmn}/\mu}a_s/2)  + \delta^k_{4} I_{n-n'\, m-m'}(-\sqrt{\rho\Omega_{lmn}/\mu}a_s/2) \nonumber  \\
&\quad - \frac{\bm{\kappa}_{mn}\cdot \hat{x}}{\rho(\omega+l\omega_d)}(\delta^k_{1} I_{n-n'\, m-m'}(|\bm{\kappa}_{mn}|a_s/2) + \delta^k_{2} I_{n-n'\, m-m'}(-|\bm{\kappa}_{mn}|a_s/2) \Big]\delta^l_{l'},  \\
A^{klmn}_{8l'm'n'}(\mathbf{k}, \omega) &= \Big[ \delta^k_{5} I_{n-n'\, m-m'}(\sqrt{\rho\Omega_{lmn}/\mu}a_s/2)  + \delta^k_{6} I_{n-n'\, m-m'}(-\sqrt{\rho\Omega_{lmn}/\mu}a_s/2)  \nonumber \\
&\quad - \frac{\bm{\kappa}_{mn}\cdot \hat{y}}{\rho(\omega+l\omega_d)}(\delta^k_{1} I_{n-n'\, m-m'}(|\bm{\kappa}_{mn}|a_s/2) + \delta^k_{2} I_{n-n'\, m-m'}(-|\bm{\kappa}_{mn}|a_s/2) \Big]\delta^l_{l'},  \\
A^{klmn}_{9l'm'n'}(\mathbf{k}, \omega) &= \Big[ \delta^k_{7} I_{n-n'\, m-m'}(\sqrt{\rho\Omega_{lmn}/\mu}a_s/2)  + \delta^k_{8} I_{n-n'\, m-m'}(-\sqrt{\rho\Omega_{lmn}/\mu}a_s/2) \nonumber \\
&\quad + \frac{i|\bm{\kappa}_{mn}|}{\rho(\omega+l\omega_d)}(\delta^k_{1} I_{n-n'\, m-m'}(|\bm{\kappa}_{mn}|a_s/2) - \delta^k_{2} I_{n-n'\, m-m'}(-|\bm{\kappa}_{mn}|a_s/2) \Big]\delta^l_{l'}. \label{lnoslip}
\end{align}

While Eqs.~\eqref{linear}-\eqref{lnoslip} form a closed linear system, it is helpful to reduce the dimension of the problem by taking advantage of the diagonality of most of the equations over the $(l,m,n)$ indices.  We eliminate all variables except for $\bm{\eta}_{lmn}=(B^x_{lmn},B^y_{lmn},H_{lmn})$ by inverting Eqs.~\eqref{lconstraint1}-\eqref{lkinematic}.

\section{Nonlinear eigenvalue algorithms}


\begin{thebibliography}{99}
\bibitem{2021_Nicolaou_1} Z. G. Nicolaou, D. J. Case, E. B. Van der Wee, M. M. Driscoll, and A. E. Motter. Heterogeneity-stabilized homogeneous states in driven media. \textit{Nature communications} \textbf{12}, 4486 (2021).
\bibitem{2021_Nicolaou_2} Z. G. Nicolaou and A. E. Motter. Anharmonic classical time crystals: A coresonance pattern formation mechanism. \textit{Physical Review Research} \textbf{3}, 023106 (2021).
\end{thebibliography}
\end{document}
